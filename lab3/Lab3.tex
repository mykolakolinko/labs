\documentclass[titlepage]{article}
\usepackage[ukrainian]{babel}
\usepackage{listings}
\usepackage{amsmath}
\usepackage{fontspec}
\setmainfont{Times New Roman}
\usepackage{setspace}
%\spacing{1.7}
\usepackage{fancyhdr}
\usepackage{titling}
\usepackage[a4paper, top = 20mm, bottom = 20mm, left = 25mm, right = 15mm]{geometry}

\lstset{
	basicstyle=\large,
	breaklines = true,
	language = matlab,
	breakatwhitespace = true,
	numbers = left,
	numberstyle = \small,
	columns = flexible,
	frame=tb,
	tabsize=4
}

\makeatletter
\renewcommand\normalsize{%
\@setfontsize\normalsize{14pt}{15pt}}
\makeatother

\preauthor{\begin{flushright}}
\postauthor{\end{flushright}}

\fancypagestyle{empty}{%
	\renewcommand{\headrulewidth}{0pt}
	\cfoot{\normalsize2016}
	\chead{\uppercase{київський національний університет імені тараса шевченка}}
}


\begin{document}

\title{\normalsize Лабораторна робота \textnumero 3}
\author{\vspace{5cm}Виконав: \protect\\ студент 4-го курсу \protect\\ спеціальність Математика \protect\\ Шатохін Михайло}
\date{}
\maketitle
\section{Постановка задачі} На відрізку $[a,b]$ методом невизначених коефіцієнтів Коллатца побудувати в точці $x_i \in \{x_j\}_{j=1}^{r+1}$ апроксимацію лінійного диференціального оператора
\[Lu(x)=u(x)+du^{(2)}(x)+ku^{(4)}(x)+u^{(m)}(x)\]
з порядком $q=r+1-m\equiv g$. Тут введено позначення:
\begin{description}
\item $[a,b] = [0,1]$,
\item $x_j=a+(j-1)h, \quad h=\frac{b-a}{r},\quad j = \overline{1,r+1}$,
\item $m=g+4,\quad i=mod(d, r+1)$,
\item $k$ --- номер за списком студента в групі,
\item $g$ --- номер групи,
\item $d$ --- день народження студента.
\end{description}
\section{Теоретичні відомості}
Для апроксимації оператора використовується метод невизначених коефіцієнтів Коллатца. Згідно з цим методом необхідно визначитися з так званим шаблоном $S(x_i)$, вузли якого і відповідні значення функції $u(x)$ використовуються у формулі числового диференціювання. Крім того, цей метод передбачає рівномірний розподіл вузлів $x_i$. За таких припущень формула числового диференціювання є зваженою сумою значень функції $u(x)$ у точках шаблона де коефіцієнти --- невідомі, а їх кількість $(r + 1)$ залежить від заданого порядку апроксимації диференціального оператора і визначиться пізніше. Ідея методу полягає у знаходженні таких значень коефіцієнтів $c_i$, щоб забезпечити найбільший порядок нев'язки. Вимагаючи певного порядку нев'язки, необхідно перш за все забезпечити в її розвиненні у ряд Тейлора нульові коефіцієнти похідних функції $u(x)$ до $m$-го порядку включно, а також додаткові рівняння для складання замкненої системи рівнянь щодо знаходження коефіцієнтів $c_i$. Тобто, усього рівнянь має бути $(r + 1)$, причому $r \ge m$. Запишемо розвинення нев'язки
$\psi(x)^2$ у ряд Тейлора в околі точки шаблона $x_i$, отримаємо систему лінійних алгебраїчних рівнянь відносно коефіцієнтів $c_i$ з такою розширеною матрицею:
\begin{equation*}
\left(\begin{array}{cccc|c}
1&1&\ldots &1&a_0\\
a_0&a_1&\ldots&a_r&\frac{a_1}{h}\\
\ldots&\ldots&\ldots&\ldots&\ldots\\
a_0^m&a_1^m&\ldots&a_r^m&m!\frac{a_m}{h^m}\\
a_0^{m+1}&a_1^{m+1}&\ldots&a_r^{m+1}&0\\
\ldots&\ldots&\ldots&\ldots&\ldots\\
a_0^{r}&a_1^{r}&\ldots&a_r^{r}&0
\end{array}\right)
\end{equation*}
матрицею Вандермонда, що не дорівнює нулю, тому що $a_k$ попарно різні. Це означає, що система лінійних алгебраїчних рівнянь відносно коефіцієнтів $c_i$ має єдиний розв'язок.
\section{Практична реалізація}
Нижче наведена реалізація методу невизначених коефіцієнтів Коллатца мовою Matlab. 
\paragraph{}
Функція main здійснює ініціалізацію параметрів задачі та виклик функції \\GetCollatzCoefficients, яка виконує основні обрахунки.
\lstinputlisting[title=main.m]{./main.m}
\paragraph{}
Функція GetCollatzCoefficients власне містить інтерпретацію мовою ./Matlab наведеного вище алгоритму невизначених коефіцієнтів Коллатца.
\lstinputlisting[title=GetCollatzCoefficients.m]{./GetCollatzCoefficients.m}
\paragraph{}
Використаний у функції main файл params.mat містить параметри задачі, що наведені в постановці задачі. Наведено текстовий варіант цього файлу з метою можливості його розуміння людиною.
\lstinputlisting[title=params.mat, language=octave]{./params.mat}
\paragraph{}
В результаті роботи даної програми отримується наступний результат
\lstinputlisting{./lab3output.txt}
\end{document}